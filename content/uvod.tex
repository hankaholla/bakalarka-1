\chapter{Introduction}
\pagenumbering{arabic}

The aim of this work is to propose a solution for detecting and classifying intracranial hemorrhage in non-contrast head CT scans, with the use of deep learning and neural networks. 

This work is divided into multiple chapters and structured as follows. After the introduction, second chapter presents the very motivation of this work, describes the medical problem of intracranial hemorrhage, its subtypes and standard process of its diagnostic. In the third chapter, we offer an overview of medical imaging, explaining its history as well as current methods of choice. Since we work with computed tomography scans, this chapter also analyzes computed tomography. Chapter number 4 represents the core of the theoretical problem analysis of this work. It is dedicated to deep learning, emphasising namely deep neural networks and layer-wise relevance propagation, since these will be used in the implementation phase of the work. Chapter 5 discusses state of the art, more specifically related works, in which the authors dealt with similar problem, and offer different approaches. Eventually, we end with summary and outlook for future work in chapter 6.

% \cite{dynabook} 
% \cite{distractions1,distractions2,interruptions}
% \cite{forgetting}
