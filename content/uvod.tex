\chapter{Introduction}
\pagenumbering{arabic}


In the recent years, artificial intelligence systems have denoted a great success. Machine learning algorithms, which are a form of artificial intelligence, have received major popularity in many areas of science and everyday life. When discussing medical field, such systems are widely used in medical imaging, including Computer-aided diagnosis, where they can help improve patients’ health care. These systems thrive due to the fact that we live in a big data era. As the knowledge and research of diseases advances, information and medical data grow accordingly. A fudamental presumption of developing a successful machine learning model in healthcare is namely an adequate amount of obtainable patients' data.


One particular task that can be addressed with machine learning approaches in medical field is detection and classification of intracranial hemorrhage from non-contrast head computed tomography examinations. Intracranial hemorrhage is a dangerous and potentially deathly condition, which calls for an early diagnostic. If not discovered promptly, the damage to the patient's brain can be severe, resulting in physical, mental, and task-based disability, or in extreme cases even death. The aim of this work is to propose a solution for for this task, with the use of deep learning and methods of computer vision. 


This work is divided into multiple chapters and structured as follows. After the introduction, second chapter presents the very motivation of this work, describes the medical problem of intracranial hemorrhage, its subtypes and standard process of its diagnostics. Chapter number three represents the core of the problem analysis of this work. It is dedicated to deep learning, emphasising namely deep neural networks and convolutional neural networks, since these will be used in the implementation phase of the work. In the fourth chapter, we offer an overview of medical imaging, explaining its history as well as current standard methods. Since we work with computed tomography scans, this chapter also analyzes computed tomography. Chapter five discusses state of the art, more specifically some related works, in which the authors dealt with similar problem and offered different approaches. In chapter six we present the concept of our solution, introduces our dataset and future work.
