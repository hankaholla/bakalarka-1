\chapter{Introduction}
\pagenumbering{arabic}


\textit{Machine learning is used in the medical imaging field, including computer-aided diagnosis, image segmentation, image registration, image fusion, image-guided therapy, image annotation,
and image database retrieval. This means that there are multiple areas in medicine, where
machine learning methods can be applied and can help improve patients’ health care [77].
For instance, machine learning methods can be used for early detection of breast cancer [78].
One particular task that can be addressed with machine learning approaches is classification,
where objects are classified (e.g. abnormal or normal, benign or malign) based upon input features [13, 74, 76]. The classification of lung diseases in computed tomography scans is one
example for the application of machine learning methods on this task [69, 74].
Deep learning methods are a set of algorithms in machine learning, which learn multiple
levels of representation and abstraction that help make sense of data [7]. Higher-level abstractions are defined from lower-level ones, so more complex functions can be learned. In particular,
if a function can be compactly represented by a deep architecture, the same function could require an extremely large architecture if the depth of this architecture is made more shallow [7].
Furthermore, this brief overview demonstrates that the algorithmic literature offers a high variety of deep learning approaches. In this thesis we will perform an empirical evaluation of
representative approaches, and discuss the conclusions from these findings. The aim of this work is to evaluate deep learning methods in the medical domain and to study if
deep learning methods outperform state of the art approaches. More precisely, the methods are
evaluated on two classification tasks, where one task consists of classifying images of a medical dataset (computed tomography images of the lung).}



The aim of this work is to propose a solution for detecting and classifying intracranial hemorrhage in non-contrast head CT scans, with the use of deep learning and methods of computer vision. 
In the recent years, medical imaging has denoted great success. Machine learning algorithms have received major popularity in many areas of science and everyday life. 
\vspace{0.8cm}
\\This work is divided into multiple chapters and structured as follows. After the introduction, second chapter presents the very motivation of this work, describes the medical problem of intracranial hemorrhage, its subtypes and standard process of its diagnostics. Chapter number three represents the core of the problem analysis of this work. It is dedicated to deep learning, emphasising namely deep neural networks and convolutional neural networks, since these will be used in the implementation phase of the work. In the fourth chapter, we offer an overview of medical imaging, explaining its history as well as current standard methods. Since we work with computed tomography scans, this chapter also analyzes computed tomography. Chapter five discusses state of the art, more specifically some related works, in which the authors dealt with similar problem and offered different approaches. In chapter six we present the concept of our solution, some first experiments and future work.
