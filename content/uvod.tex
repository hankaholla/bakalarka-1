\chapter{Introduction}
\pagenumbering{arabic}

The aim of this work is to propose a solution for detecting and classifying intracranial hemorrhage in non-contrast head CT scans, with the use of deep learning and methods of computer vision. 
In the recent years, medical imaging has denoted great success. Machine learning algorithms have received major popularity in many areas of science and everyday life. 
\vspace{0.8cm}
\\This work is divided into multiple chapters and structured as follows. After the introduction, second chapter presents the very motivation of this work, describes the medical problem of intracranial hemorrhage, its subtypes and standard process of its diagnostics. Chapter number three represents the core of the problem analysis of this work. It is dedicated to deep learning, emphasising namely deep neural networks and convolutional neural networks, since these will be used in the implementation phase of the work. In the fourth chapter, we offer an overview of medical imaging, explaining its history as well as current standard methods. Since we work with computed tomography scans, this chapter also analyzes computed tomography. Chapter five discusses state of the art, more specifically some related works, in which the authors dealt with similar problem and offered different approaches. In chapter six we present the concept of our solution, some first experiments and future work.
