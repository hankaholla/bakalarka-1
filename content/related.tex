\chapter{Related work}
In this chapter I would like to introduce some related works, in which the authors propose different approaches to the problem of detection and classification of intracranial hemorrhage in non-contrast CT scans.

\section{Identification of intracranial hemorrhage with clinical workflow integration}
 
In their work \cite{relatedWork1} Arbabshirani et al. developed a machine learning algorithm, which automatically analyzed head CT scans, flagging those with the presence of a hemorrhage, and thus prioritized radiology work lists in real time. As a result, the average time, in which a patient received their diagnosis, had been significantly reduced.  

The machine learning algorithm was implemented as a three-dimensional convolutional neural network, which had five convolutional layers and two fully connected layers. The dataset used in development of the solution contained 46 583 non-contrast head CT studies, each having 20 axial two-dimensional slices. Each study had been manually labeled with a binary value, signalizing the presence of a hemorrhage. Prior to the network training, the data had been preprocessed with several techniques, such as standardizing the number of axial slices and resizing each slice, so to achieve a uniform dimensions of 256 x 256 x 24, as well as applying windowing, in order to increase the contrast in each of the slices. Thenceforth, the neural network was trained using stochastic gradient descent, until the received loss was near zero. After the training period, the algorithm was tested on previously unseen test data.

Over a three-month period, the trained model had been implemented at a clinic in Pennsylvania, USA to assist with prioritization of radiology work lists. During this implementation phase, 347 head CT studies were processed in real time. For each study, which was fed to the algorithm, a binary output was produced, representing positive or negative presence of ICH. Studies marked as positive were assigned higher priority and moved up the radiology work list. Out of these 94 studies, 60 were determined by a interpreting radiologist as true positive, which gave the model positive predictive value of 64\%. were consequently . With this approach, average diagnostic time was shortened from 8,5 hours to just 19 minutes, which is acceleration of 96\%. The algorithm showed accuracy of 84\% sensitivity of 70\%  and specificity of 87\%. The area under the curve of the model, was established to be 0.846.

\begin{figure}[h]
\begin{centering}
\includegraphics[width=16cm]{assets/images/RW1-net-arch.png}
\par\end{centering}
\caption{Model architecture from \cite{relatedWork1} \label{fig:rw1}}
\end{figure}


\section{Druhá práca}
\lipsum